\newpage
\section{Introduction}
Rational Expectations Equilibrium (REE) requires that economic agents' subjective probability distributions coincide
with the objective distribution that is determined, in part, by
their subjective beliefs. There is a vast literature that studies
the drawbacks of REE. Some of these drawbacks include the fact that
REE requires an unrealistic degree of computational power and perfect
information on the part of agents. Alternatively, the adaptive
learning literature (see, e.g., Evans and Honkapohja (2001, 2013)
and Bullard (2006) for extensive surveys and references) replaces
Rational Expectations with beliefs that come from an econometric
forecasting model with parameters updated using observed time
series. A large part of this literature involves studying under
which conditions learning will converge to the rational expectations
equilibrium. When the perceived law of motion (PLM) of agents is
correctly specified, convergence of adaptive learning to an REE can
occur. However, in general the PLM may be misspecified. As shown in White
(1994), an economic model or a probability model is only
a more or less crude approximation to whatever might be the true
relationships among the observed data. Consequently it is
necessary to view economic and/or probability models as misspecified
to some greater or lesser degree. Whenever agents have {\it misspecified}
PLMs, a reasonable learning process may settle down to a misspecification equilibrium. In the literature, different
types of misspecification equilibria have been proposed, e.g. Restricted Perceptions Equilibrium (RPE) where the forecasting model is underparameterized (Sargent, 1991; Evans and Honkapohja, 2001; Adam, 2003; Branch and Evans, 2010) and Stochastic Consistent Expectations Equilibrium (SCEE) (Hommes and Sorger, 1998; Hommes et al., 2013), where agents learn the optimal parameters of a simple, parsimonious AR(1) rule.\footnote{Branch (2006) provides a stimulating survey discussing the connection between these types of misspecification equilibria.}

A SCEE is a very natural misspecification equilibrium, where agents in the economy do not know the actual law of motion or even recognize all relevant explanatory variables, but rather prefer a parsimonious forecasting model. The economy is too complex to fully understand and therefore, as a first-order approximation, agents forecast the state of the economy by simple autoregressive models (e.g. Fuster et al., 2010).
In the simplest model applying this idea, agents run a univariate AR(1) regression to generate out-of-sample forecasts of the state of
the economy. Hommes and Zhu (2014) provide the first-order SCEE with an \emph{intuitive behavioral}
interpretation and refer to them as  {\it Behavioral Learning Equilibria} (BLE). Although it is possible for some agents to use more sophisticated
models, one may argue that these practices are neither straightforward nor widespread. A simple, parsimonious BLE seems a more plausible outcome of the coordination process of individual expectations in large complex socio-economic systems (Grandmont, 1998).

Hommes and Zhu (2014) formalize the concept of BLE in the simplest class of models one can think of: a one-dimensional linear stochastic model driven by an exogenous linear stochastic AR(1) process. Agents do not recognize, however, that the economy is driven by an exogenous AR(1) process, but simply forecast the state of the economy using a
univariate AR(1) rule. The parameters of the AR(1) forecasting rule are not free, but fixed (or learned over time) according to the observed sample average and first-order sample autocorrelation. Within this simple but general class of models, Hommes and Zhu (2014) fully characterize the existence and multiplicity of BLE and provide stability conditions under a simple adaptive learning scheme --Sample Autocorrelation Learning (SAC-learning). Although this class of models is simple, it contains two important standard applications: an asset pricing model driven by autocorrelated dividends and the New Keynesian Phillips curve with inflation driven by autocorrelated output gap (or marginal costs). As shown in Fuhrer (2009), however, the skeleton model of the New Keynesian Phillips curve with AR(1) driving variable leaves implicit the determination of real output and the role of monetary policy in influencing output and inflation.

In this paper we extend the BLE concept to a general n-dimensional linear stochastic framework and provide a method to estimate BLE in this framework. As an application we consider the standard 3-equation dynamic stochastic general equilibrium (DSGE) model-the New Keynesian (NK) model-, study the empirical fit of the model and the role of monetary policy under BLE. Agents' perceived law of motion (PLM) is a simple univariate AR(1) process for each variable to be forecasted. Two consistency requirements are imposed upon BLE to pin down the parameters of the forecasting model: for each endogenous variable observed sample averages and first-order sample autocorrelations match the corresponding parameters of the forecasting rule. Agents thus learn the optimal AR(1) forecasting rule for each endogenous variable in the economy.

Numerous empirical studies show that overly parsimonious models with little parameter uncertainty can provide better forecasts than models consistent with the actual data-generating complex process (e.g. Nelson, 1972; Stock and Watson, 2007; Clark and West, 2007; Enders, 2010). In a similar vein (but without analytical results) Slobodyan and Wouters (2012) study a New Keynesian DSGE model with agents using a constant gain AR(2) forecasting rule. Chung and Xiao (2014) and Xiao and Xu (2014) study learning and predictions with an AR(1) or VAR(1) model in a two dimensional New Keynesian model with limited information and show, based on simulations, that the simple AR(1) model is more likely to prevail in reality when they make predictions. Laboratory experiments in the NK framework also show that simple forecasting rules such as AR(1) describe individual forecasting behavior surprisingly well (Assenza et al., 2014; Pfajfar and Zakelj, 2016).

Our paper bears clear resemblance to the seminal work of Krusell and Smith (1998), where the behavior of macroeconomic aggregates can be almost perfectly described by the mean of the wealth distribution and the aggregate productivity shock. In our BLE agents are boundedly rational and forecast macro variables by learning the mean and the first-order autocorrelations of all observable endogenous macro variables. Such irrationality is hard to detect however, as agents learn  to forecast correctly the mean and the first-order autocorrelation of each variable in the economy. In fact, for each endogenous variable the optimal univariate rule satisfies the orthogonality condition for rational expectations. 

The main contributions of our paper are fourfold: (1) we derive existence and stability conditions of BLE in a general linear framework, (2) we provide a simple and general method for Bayesian likelihood estimation of BLE, (3) we estimate the baseline NK model based on U.S. data and show that the relative model fit is better under BLE than REE, (4) we analyze the optimal monetary policy under BLE and compare it with REE.

%proofs of BLE in a general linear framework, (2) stability conditions of BLE under sample autocorrelation (SAC-) learning, (3) persistence and volatility amplification, and %(4) monetary policy stabilization analysis under BLE.(4) optimal monetary policy analysis under BLE.
Many models of learning lead to excess volatility, where the volatility under learning is typically higher than under REE. Our BLE model exhibits another novel feature, {\it persistence amplification}: the persistence of inflation and output gap under BLE is significantly higher than under REE. In fact, even when autocorrelations of the exogenous shocks to fundamentals are small, inflation and output gap along BLE are typically near unit root processes. As a consequence, 
when we estimate the NK model under BLE, we find important differences in parameter estimates compared with the REE. We further analyze optimal monetary policy under BLE and find finite optimal Taylor rule coefficients under a wide range of calibrations. Further, we find that transmission channel of monetary policy is stronger under BLE at the estimated parameters.
 %Furthermore, the finite optimal policies under BLE have different responses to the persistence of shocks across the different Taylor rules.


\subsection*{Related literature}

The issue of persistence has been of great interest to macroeconomists and policymakers. A number of models with frictions have been proposed to replicate persistence, such as habit formation in consumption, indexation to lagged inflation in price-setting, rule-of-thumb behavior, or various adjustment costs (Phelps, 1968; Taylor, 1980; Fuhrer and Moore, 1992, 1995; Christiano et al., 2005; Smets and Wouters, 2003, 2005; Boivin and Giannoni, 2006; Giannoni and Woodford, 2003). These models essentially improve the empirical fit by adding lags in the model equations. Estimating these rich models with frictions under the assumption of rational expectations, one typically finds that substantial degrees of persistence are supported by the data. Therefore these additional sources of persistence appear necessary to match the inertia of macroeconomic variables. Estimation of these models typically also involve highly persistent structural shocks. Our BLE model is applied to a New Keynesian framework without habit formation or indexation, but nevertheless exhibits strong persistence. Learning causes persistence amplification: small autocorrelations of exogenous shocks are strongly amplified as agents learn to coordinate on a simple AR(1) forecasting rule with near unit root parameters consistent with observed sample average and sample autocorrelations. The high persistence of inflation and output thus arises from a self-fulfilling mistake (Grandmont, 1998).

Our BLE concept fits with the literature employing adaptive learning to analyze the evolution of U.S. inflation and monetary policy. Adaptive learning can help in understanding some particular historical episodes, such as high inflation in the 1980s, which are often harder to explain under rational expectations. For example, Orphanides and Williams (2003) consider a form of imperfect knowledge in which economic agents rely on adaptive  learning to form expectations. This form of learning represents a relatively modest deviation from rational expectations that nests it as a limiting case. They find that policies that would be efficient under rational expectations can perform poorly when knowledge is imperfect. Milani (2005, 2007) also assumes that agents form expectations through adaptive learning using correctly specified economic models and updating the parameters through constant-gain learning (CGL) based on historical data. He shows empirically that when learning replaces rational expectations, the estimated degrees of habits and indexation drop closer to zero, suggesting that persistence arises in the model economy mainly from expectations and learning. Eusepi and Preston (2011) study expectations-driven business cycles based on learning, and find that learning dynamics generate forecast errors similar to the Survey of Professional Forecasters. Estrella and Fuhrer (2002) study the shortcomings of REE models with a focus on inertia and shock propagation structure.
Fuhrer (2009) provides a good survey on inflation persistence. He examines a number of empirical measures of reduced form persistence including the first-order autocorrelation and the autocorrelation function of the inflation series. He also investigates the sources of persistence, including learning of agents in a rational- expectation setting.

Our behavioral learning equilibrium concept is closely related to the Exuberance Equilibria (EE) in Bullard et al. (2008), where agents' perceived law of motion is misspecified. However, because of difficulty of computation, in Bullard et al. (2008) there are only numerical results on the exuberance equilibria, while here we analytically show the existence and stability of BLE in a general linear framework with an application to the NK model, as well as empirically validate BLE based on U.S. data. Another related misspecification  equilibrium is Limited Information Learning Equilibrium (LILE) defined in Chung and Xiao (2014), which is defined by the least-squares projection of variables on the past information of the actual law of motion equal to that in the perceived law of motion. Different from the LILE, our general Behavioral Learning Equilibrium is defined by the conditions that sample means and first-order autocorrelations of each variable of the actual law of motion are consistent with those corresponding to the perceived law of motion. We further study the effects of monetary policy under the more plausible BLE. The concept of natural expectations in Fuster et al. (2010) and Fuster et al. (2011, 2012) is another misspecification concept, where agents use simple, misspecified models, e.g., linear autoregressive models. Natural expectations, however, do not pin down the parameters of the forecasting model through consistency requirements as for a restricted perceptions equilibrium nor do they allow the agents to learn an optimal misspecified model through empirical observations. Cho and Kasa (2015) study model validation in an environment where agents are aware of misspecification and try to detect it through adaptive learning. In our BLE misspecification is self-fulfilling and the outcome of the SAC-learning process.

The paper is organized as follows. Section 2 introduces the main concepts of BLE in a general n-dimensional setup, the theoretical results on existence and stability of BLE in a linear framework, and the empirical estimation methodology. Section 3 applies BLE to the 3-equation New Keynesian model and studies the existence, stability and estimation results under BLE. Section 4 studies optimal monetary policy and how policy can mitigate persistence and volatility amplification under BLE. Finally, Section 5 concludes.


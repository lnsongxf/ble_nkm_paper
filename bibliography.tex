\begin{thebibliography}{}

\bibitem{A03} Adam, K., 2003.
Learning and equilibrium selection in a monetary overlapping generations model with sticky prices.
{Review of Economic Studies} 70, 887-907.

\bibitem{A07} Adam, K., 2007. Experimental evidence on the persistence of output and inflation. Economic Journal 117, 603-636.

\bibitem{} Adjemian, S., Bastani, H., Juillard, M., Mihoubi, F., Perendia, G., Ratto, M., \& Villemot, S. 2011. Dynare: Reference manual, version 4.

\bibitem{} An, S., \& Schorfheide, F. 2007. Bayesian analysis of DSGE models. Econometric reviews, 26(2-4), 113-172.


\bibitem{AHHM12}Assenza, T., Heemeijer, P., Hommes, C. and Massaro, D., 2014.
Individual Expectations and Aggregate Macro Behavior. CeNDEF Working Paper, University of Amsterdam.

\bibitem{bh14} Boehm, C.E., House, C.L., 2014. Opitmal Taylor rules in New keynesian models. NBER Working Paper Series: w20237.

\bibitem{ahhm10} Boivin, J., Giannoni, M., 2006. Has monetary policy become more
effective? The Review of Economics and Statistics 88 (3), 445-462.

%
%\bibitem{} Branch, W.A., 2004. The theory of rationally
%heterogeneous expectations: evidence from survey data on inflation
%expectations. Economic Journal 114, 592-621.
%
\bibitem{B06} Branch, W.A., 2006. Restricted perceptions equilibria and learning in macroeconomics, in: Colander, D.
(Ed.), Post Walrasian Macroeconomics: Beyond the Dynamic Stochastic
General Equilibrium Model. Cambridge University Press, New York, pp.
135-160.

\bibitem{} Branch, W.A., Evans, G.W., 2010. Asset return dynamics and learning. The Review of Financial Studies 23 (4), 1651-1680.

\bibitem{} Bray, M., 1982. Learning, estimation, and the stability of rational expectations. Journal of economic theory, 26(2), pp.318-339.

\bibitem{} Brock, W.A., Malliaris, A.G., 1989. Differential Equations, Stability and Chaos in Dynamic Economics. North-Holland, Amterdam.
%
%\bibitem{}Branch, William A., 2007. Sticky information and model uncertainty in survey
%Data on inflation expectations. Journal of Economic Dynamics
%\& Control 31 (1), 245-276.
%
%\bibitem{} Branch, W.A., Evans, G.W., 2011. Monetary policy and heterogeneous expectations. Economic Theory 47, 365-393.
%
%
%\bibitem{}Branch, W.A., McGough, B., 2009. A New Keynesian model with heterogeneous expectations. Journal of Economic Dynamics
%and Control 33, 1036-1051.

%\bibitem{BM05} Branch, W.A., McGough, B., 2005. Consistent expectations and misspecification in
%stochastic non-linear economies. Journal of Economic Dynamics \&
%Control 29, 659-676.

%\bibitem{BH97}Brock, W.A., Hommes, C.H., 1997. A rational route to
%randomness. Econometrica 65, 1059-1095.
%%
%\bibitem{BH98} Brock, W., Hommes, C., 1998. Heterogeneous beliefs and routes to
%chaos in a simple asset pricing model. Journal of Economic Dynamics
%and Control 22, 1235-1274.

%\bibitem{BD91} Brockwell, P.J., Davis, R.A., 1991. Time Series:
%Theory and Methods. Springer-Verlag, New York.




\bibitem{B06} Bullard, J., 2006. The learnability criterion and monetary
policy. Federal Reserve Bank of St. Louis Review 88, 203-217.

%
\bibitem{BEH08} Bullard, J., Evans, G.W., Honkapohja, S., 2008.
Monetary policy, judgment and near-rational exuberance. American
Economic Review 98, 1163-1177.

%\bibitem{BEH10} Bullard, J., Evans, G.W., Honkapohja, S., 2010.
%A model of near-rational exuberance. Macroeconomic Dynamics 14,
%166-188.
%
\bibitem{BM02} Bullard, J., Mitra, K., 2002.
Learning about monetary policy rules. Journal of Monetary Economics
49, 1105-1129.

\bibitem{} Caplin, A. and Leahy, J., 1996.
 Monetary policy as a process of search. The American Economic Review, pp.689-702.

\bibitem{ChoKasa2015}
Cho, I.K., Kasa, K., 2015.
Learning and model validation.
{Review of Economic Studies} 82, 45-82.


\bibitem{} Christiano, L. J., Eichenbaum, M., \& Evans, C. L., 1999.
 Monetary policy shocks: What have we learned and to what end?. Handbook of macroeconomics, 1, 65-148.

\bibitem{} Chung, H., Xiao, W., 2014.
Cognitive consistency, signal extraction and macroeconomic persistence.
Working Paper Binghamton University.

\bibitem{} Clarida, R., Gal\'{i}, J., Gertler, M., 1999.
The science of monetary policy: a New Keynesian perspective. {Journal of Economic Literature} 37, 1661-1707.

\bibitem{CW07} Clark, T., West, K., 2007.
Approximately normal tests for equal predictive accuracy in nested models. Journal of Econometrics
138, 291-311.

\bibitem{}Christiano, L.J., Eichenbaum, M., Evans, C.L., 2005.
Nominal rigidities and the dynamic effects of a shock to monetary policy. Journal of Political Economy 113 (1).

\bibitem{} DeCanio, S.J., 1979. Rational expectations and learning from experience. The quarterly journal of economics, 93(1), pp.47-57.

\bibitem{E99} Elaydi, S.N., 1999. An Introduction to Difference
Equations, 2nd edition. Springer, New York.

\bibitem{E10}Enders, W., 2010. Applied Econometric Time Series (3rd ed.). John Wiley \& Sons, Inc., USA.

\bibitem{estrellafuhrer} Estrella, A., \& Fuhrer, JC, 2002.
Dynamic inconsistencies: Counterfactual implications of a class of rational-expectations models. 
{American Economic Review} ,92(4), 1013-1028.

\bibitem{eusepiexpectations} Eusepi, Stefano, and Bruce Preston, 2011.
Expectations, learning, and business cycle fluctuations. {American Economic Review} 101.6: 2844-72.

\bibitem{} Evans, G., 1985. Expectational stability and the multiple equilibria problem in linear rational expectations models. The Quarterly Journal of Economics, 100(4), pp.1217-1233.

\bibitem{EH01}Evans, G.W., Honkapohja, S., 2001. Learning and Expectations in Macroeconomics. Princeton University Press, Princeton.


\bibitem{}Evans, G.W., Honkapohja, S., 2003. Expectations and the stability problem for optimal monetary policies. The Review of
Economic Studies 70, 807-824.

\bibitem{EH01}Evans, G.W., Honkapohja, S., 2013. Learning as a rational foundation for macroeconomics and finance.
In: Roman Frydman and Edmund S. Phelps (Eds.), Rethinking Expectations: The Way Forward for
Macroeconomics (Chapter 2). Princeton
University Press.

\bibitem{} Farmer, R. E., Waggoner, D. F., \& Zha, T. 2009. Understanding Markov-switching rational expectations models. Journal of Economic theory, 144(5), 1849-1867.

\bibitem{} Fuhrer, J., Moore, G., 1992. Monetary policy rules
and the indicator properties of asset prices. Journal of Monetary
Economics 29, 303-336.

\bibitem{FM95} Fuhrer, J., Moore, G., 1995. Inflation persistence. Quarterly
Journal of Economics 110 (1), 127-159.
%
\bibitem{F06}Fuhrer, J.C., 2006. Intrinsic and inherited inflation persistence. International Journal of Central Banking 2, 49-86.
%
\bibitem{F09}Fuhrer, J.C., 2009. Inflation persistence. Federal Reserve Bank of Boston, working
paper.



\bibitem{}
Fuster, A., Hebert, B., Laibson, D., 2011. Natural expectations,
macroeconomic Dynamics, and asset pricing. Forthcoming NBER
Macroeconomics Annual 26.
\bibitem{}
Fuster, A., Hebert, M., Laibson, D., 2012. Investment Dynamics with
Natural Expectations. International Journal of Central Banking
(Special Issue in Honor of Benj










\bibitem{FusLaiMen10}
Fuster, A., Laibson, D. and Mendel, B., 2010. Natural expectations
and macroeconomic fluctuations. Journal of Economic Perspectives 24,
67-84.

%\bibitem{GGL01}Gali, J., Gertler, M., L\'{o}pez-Salido, J.D., 2001. European inflation dynamics. European Economic Review 45, 1237-1270.

\bibitem{Gal2008}
Gal\'{i}, J., 2008. Monetary Policy, Inflation, and the Business Cycle: An Introduction to the New Keynesian Framework. Princeton University Press, New Jersey.

\bibitem{}Giannoni, M., Woodford, M., 2003. Optimal inflation targeting rules.
In: Bernanke, B.S., Woodford, M. (Eds.), Inflation Targeting.
University of Chicago Press, Chicago.

\bibitem{}Giannoni, M.P., 2014. Optimal interest-rate rules and inflation stabilization versus price-level stabilization. 
Journal of Economic Dynamics and Control, 41, pp.110-129.

\bibitem{Gra98}
Grandmont, J.M., 1998. Expectation formation and stability in large
socio-economic systems. Econometrica 66, 741-781.

\bibitem{} Greenberg, E. 2012. Introduction to Bayesian econometrics. Cambridge University Press



\bibitem{H94}Hamilton, J.D., 1994. Time Series Analysis. Princeton
Univeristy Press.

\bibitem{hartigan1985dip} Hartigan, John A and Hartigan, PM, 1985.
The dip test of unimodality.
{The Annals of Statistics} ,70-84.

\bibitem{} Herbst, E. P., \& Schorfheide, F. 2015. Bayesian estimation of DSGE models. Princeton University Press.


\bibitem{} Horn, R. A., \& Johnson, C. R. 1985. Matrix analysis cambridge university press. New York, 37.
%
%\bibitem{HL07JEDC} He, X., Li, Y., 2007. Power law behaviour, heterogeneity, and trend chasing. \emph{Journal
%of Economic Dynamics and Control} 31, 3396-3426.
%
%\bibitem{H06} Hommes, C., 2006. Heterogeneous agent models in economics and
%finance, in: Tesfatsion, L., Judd, K.L. (Eds.), Handbook of
%Computational Economics, Vol. 2: Agent-Based Computational
%Economics. Elsevier Science B.V.
%
%\bibitem{H11} Hommes, C.H., 2011. The heterogeneous expectations
%hypothesis: some evidence for the lab. Journal of Economic Dynamics
%\& Control 35, 1-24.
%%
\bibitem{HS98} Hommes, C., Sorger, G., 1998. Consistent expectations
equilibria. Macroeconomic Dynamics 2, 287-321.
%
%\bibitem{HR01} Hommes, C., Rosser, J.B., 2001. Consistent expectations equilibria and complex
%dynamics in renewable resource markets. Macroeconomic Dynamics 5,
%180-203.
%%

\bibitem{HSW13} Hommes, C.H., Sorger, G., Wagener, F., 2013.
Consistency of linear forecasts in a nonlinear stochastic economy.
In: Bischi, G.I., Chiarella, C. and Sushko, I. (Eds.), {\it Global
Analysis of Dynamic Models in Economics and Finance},
Springer-Verlag Berlin Heidelberg, pp. 229-287.

\bibitem{HZ} Hommes, C., Zhu, M., 2014. Behavioral Learning Equilibria. Journal of Economic Theory 150, 778-814.

\bibitem{KruSmi98} Krusell, P. and Smith, A., 1998. Income and wealth heterogeneity in the macroeconomy, {\it Journal of Political Economy} 106, 867-896.

%\bibitem{L72} Lucas, R., 1972. Expectations and the neutrality of moeny. Journal of Economic Theory 4, 103-124.
%
%\bibitem{M07} Milani, F., 2007. Expectations, learning and macroeconomic persistence. Journal of Monetary Economics 54, 2065-2082.
%%
%\bibitem{M61}Muth, J.E., 1961. Rational expectaions and the theory of price movements. Ecoometrica 29, 315-335.
%
%%

\bibitem{} Lancaster, P., Tismenetsky, M., 1985. The Theory of Matrices (Second Edition with Applications). Academic Press, San Diego.


\bibitem{} Leeper, E. M., Sims, C. A., Zha, 1998. What does monetary policy do?{\it Brookings papers on economic activity}, 1-78.
%\bibitem{L09}Lansing, K.J., 2009. Time-varing U.S. inflation dynamics and the new Keynesian Phillips curve. Review of Economic Dynamics 12, %304-326.

%\bibitem{} Kurz, M., 2011. A New Keynesian model with diverse beliefs. Working paper, Stanford University.
%
%\bibitem{} Mankiw, N.G., Reis, R., Wolfers, J., 2003. Disagreement
%about inflation expectatoins. NBER Macroeconomics Annual 18,
%209-248.
\bibitem{} Magnus, J., Neudecker, H., 1988. Matrix Differential calculus. Wiley, New York.

\bibitem{M99} Martelli, M., 1999. Introduction to Discrete Dynamical
Systems and Chaos.  Wiley, New York.


\bibitem{} Milani, F. 2005. Adaptive learning and inflation persistence. University of California, Irvine-Department of Economics.


\bibitem{M07} Milani, F., 2007. Expectations, learning and
macroeconomic persistence. Journal of Monetary Economics 54,
2065-2082.



\bibitem{N72}Nelson, C., 1972. The prediction performance of the FRB-MIT-PENN model of the US economy. American Economic Review 62, 902-917.

\bibitem{office2001cbo} {Office, Congressional Budget and Congress, US. 2001.
CBO’s Method for Estimating Potential Output: An Update.
{August (Washington, DC: Congressional Budget Office)}.



%
\bibitem{} Orphanides, A., Williams, J., 2003.
Imperfect knowledge, inflation expectations and monetary policy.
In: Bernanke, B., Woodford, M. (Eds.), Inflation Targeting. University of Chicago Press, Chicago.
%
\bibitem{} Phelps, E. S., 1968. Money-wage dynamics and labor-market
equilibrium. Journal of Political Economy 76(4, Part 2), 678-711.
%
\bibitem{} Pfajfar, D., $\check{{Z}}$akeljz, B., 2016. Inflation expectations and monetary policy design:
 evidence from the laboratory. Forthcoming in Macroeconomic Dynamics.

\bibitem{S91}Sargent, T.J., 1991. Equilibrium with signal extraction from endogenous variables. Journal
of Economic Dynamics \& Control 15, 245-273.



%\bibitem{S93} Sargent T.J., 1993. Bounded Rationality in Macroeconomics. Oxford University Press Inc., New York.
%
%\bibitem{S99} Sargent T.J., 1999. The Conquest of American Inflation. Princeton University Press, Princeton, NJ.

\bibitem{} Slobodyan, S., Wouters, R., 2012.
Learning in a medium-scale DSGE model with expectations based on small forecasting models.
{American Economic Journal: Macroeconomics} 4, 65-101.

\bibitem{} Smets, F., Wouters, R., 2003. Monetary policy in an estimated
stochastic dynamic general equilibrium model of the euro area.
Journal of the European Economic Association 1 (5), 1123-1175.

\bibitem{} Smets, F., Wouters, R., 2005. Comparing shocks and frictions in US
and euro business cycles: a Bayesian DSGE approach. Journal of
Applied Econometrics 20 (2), 161-183.

\bibitem{StoWat07}
Stock, J.H., Watson, M.W., 2007. Why has inflation become harder
to forecast? {Journal of Money, Credit and Banking} 39, 3-34.



%
%%\bibitem{SM02} S\"{o}gner, L., Mitl\"{o}hner, H., 2002. Consistent expectations equilibria and learning in a stock
%%market. Journal of Economic Dynamics \& Control 26, 171-185.
%
%
%\bibitem{T03} Tallman, E. W., 2003. Monetary policy and learning:
%some implications for policy and research. Federal Reserve Bank of
%Atlanta, ECONOMIC REVIEW, Third Quater.
%
%
\bibitem{} Taylor, J., 1980. Aggregate dynamics and staggered
contracts. Journal of Political Economy 88, 1-23.

%\bibitem{T03} Tuinstra, J., 2003. Beliefs equilibria in an overlapping generations
%model. Journal of Economic Behavior \& Organization 50, 145-164.

\bibitem{W94} White, H., 1994. Estimation, Inference and Specification
Analysis. Cambridge University Press, Cambridge.

\bibitem{} Woodford, M., 1999. Optimal monetary policy inertia. The Manchester School, 67, pp.1-35.

\bibitem{W03} Woodford, M., 2003. Interest and Prices. Princeton University Press,
Princeton.



\bibitem{} Xiao, W., Xu, J. 2014.
Expectations and optimal monetary policy: a stability problem revisited.
{Economics Letters} 124, 296-299.





%\bibitem{} Slobodyan, S., & Wouters, R. 2012. Learning in an estimated medium-scale DSGE model. Journal of Economic Dynamics and control, 36(1), 26-46.


%\bibitem{} Smets, F., & Wouters, R. 2007. Shocks and frictions in US business cycles: A Bayesian DSGE approach. American economic review, 97(3), 586-606.

%\bibitem{methods1998dynamic} Sims, Christopher A and Zha, Tao, 1998.
%Bayesian Methods for Dynamic Multivariate Models.
%{International Economic Review} 150, 949-968.



%\bibitem{hirose2016zero} Hirose, Yasuo and Sunakawa, Takeki, 2016.
%The zero lower bound and parameter bias in an estimated DSGE model.
%{Journal of Applied Econometrics} 31, 4, 630-651.

%%%%%%%%\bibitem{bullard2002learning} Bullard, James and Mitra, Kaushik. 2002.
%%%%%%%Learning About Monetary Policy Rules.
%%%%%%{Journal of Monetary Economics} 49,6,  296-299.

%%%%%\bibitem{hommes1998consistent} Hommes, Cars and Sorger, Gerhard. 1998.
%%%%Consistent Expectations Equilibria.
%%%{Macroeconomic Dynamics} 2,3,  287-321.

%%\bibitem{fuhrer1995inflation} Fuhrer, Jeff and Moore, George. 1995.
%Inflation Persistence.
%{The Quarterly Journal of Economics} 110,1,  127-159.

%\bibitem{fuster2010natural} Fuster, Andreas and Laibson, David and Mendel, Brock. 2010.
%Natural Expectations and Macroeconomic Fluctuations.
%{The Journal of Economic Perspectives} 24,4,  67-84.

%\bibitem{smets2003estimated} Smets, Frank and Wouters, Raf. 2003.
%An Estimated Dynamic Stochastic General Equilibrium Model of the Euro Area.
%{Journal of the European Economic Association} 1,5,  1123-1175.

%\bibitem{smets2007shocks} Smets, Frank and Wouters, Raf. 2007.
%Shocks and Frictions in US Business Cycles: A Bayesian DSGE Approach.
%{The American Economic Review} 97,3,  586-606.

%\bibitem{bihan2012sticky} Bihan, Herv{\'e} Le and Montorn{\`e}s, J{\'e}r{\'e}mi and Heckel, Thomas. 2012.
%Sticky Wages: Evidence from Quarterly Microeconomic Data.
%{American Economic Journal: Macroeconomics} 4,3,  1-32.

%\bibitem{slobodyanJEDC} Slobodyan, Sergey and Wouters, Raf. 2012.
%Learning in an estimated medium-scale DSGE model.
%{Journal of Economic Dynamics and control} 36,1,  26-46.

%\bibitem{slobodyan2012learning} Slobodyan, Sergey and Wouters, Raf. 2012.
%Learning in a Medium-scale DSGE Model with Expectations Based on Small Forecasting Models.
%{American Economic Journal: Macroeconomics} 4,2,  65-101.

%\bibitem{heckel2008sticky} Heckel, Thomas and Le Bihan, Herv{\'e} and Montorn{\`e}s, J{\'e}r{\'e}mi}. 2008.
%Sticky wages: evidence from quarterly microeconomic data.
%{}.

%\bibitem{giannoni2004optimal} Woodford, Michael. 2003.
%Interest and Prices.
%{Princeton University Press}.


%\bibitem{taylor1999historical} Taylor, John B. 1999.
%A Historical Analysis of Monetary Policy Rules.
%{University of Chicago Press}.

%\bibitem{judd1998taylor} Judd, John P and Rudebusch, Glenn D. 1998.
%Taylor's Rule and the Fed: 1970-1997.
%{Economic Review-Federal Reserve Bank of San Francisco} 3,3.

%\bibitem{cooley1992tax} Cooley, Thomas F and Hansen, Gary D. 1992.
%Tax Distortions in a Neoclassical Monetary Economy.
%{Journal of Economic Theory} 58,2,  290-316.



%\bibitem{schmitt2005optimal} Schmitt-Groh{\'e}, Stephanie and Uribe, Martin. 2005.
%Optimal Fiscal and Monetary Policy in a Medium-scale Macroeconomic Model.
%{NBER Macroeconomics Annual} 20,  383-425.

%\bibitem{cooley1992tax} Cooley, Thomas F and Hansen, Gary D. 1992.
%Tax Distortions in a Neoclassical Monetary Economy.
%{Journal of Economic Theory} 58,2,  290-316.

}

\end{thebibliography}
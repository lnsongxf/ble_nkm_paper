\section{Concluding Remarks}
We have generalized the behavioral learning equilibrium concept to an n-dimensional linear stochastic framework and provided a general iterative method to approximate and estimate BLE.
We have applied our framework to the 3-equation New Keynesian model. Boundedly rational agents use univariate AR(1) forecasting rules for all output gap and inflation. A BLE is parameter free, as along the BLE the two parameters of each rule are pinned down by two observable statistics: the unconditional mean and the first-order
autocorrelation. Hence, to a first-order approximation the simple linear forecasting rule is optimal and consistent with observed market realizations. Agents gradually update the two coefficients --sample mean and
first-order autocorrelation-- of their linear rule through sample autocorrelation learning. In the long run, agents thus learn to coordinate on the best univariate linear forecasting rule for each endogenous state variable, without fully recognizing the more complex structure of the economy.  In higher-dimensional systems, BLE exist under fairly general conditions and we provide simple stability conditions under learning. Coordination on a simple, parsimonious BLE is self-fulfilling and seems a plausible outcome of the coordination process of individual expectations in large complex socio-economic systems.

A striking feature of BLE is the strong {\it persistence amplification}: the persistence of output and inflation along a BLE is much higher, often near unit root, than the persistence in the exogenous shocks driving the economy. Due to these features, estimating the 3-equation model on historical data under BLE yields a substantially better model fit and a different propagation mechanism compared with the REE model.
This leaves an important role for monetary policy with the goal of stabilizing inflation and output.  Different from REE, we find finite optimal Taylor rule coefficients at the BLE in our benchmark calibration. Furthermore, we observe a stronger transmission channel of monetary policy at the estimated parameter values under BLE. A sufficiently aggressive Taylor rule may keep the economy in a stable region with relatively low volatility in inflation and output gap. Future work should study BLE and corresponding optimal policies in more general New Keynesian models.
%We study monetary policies with a Taylor interest rate rule. There are strong direct effects: more aggressive inflation (output) targeting weakens the persistence in inflation (output). Indirect effects may be destabilizing however: more aggressive inflation (output) targeting may lead to more persistent output (inflation). To stabilize both inflation and output, monetary policy must therefore carefully balance between inflation and output targeting. More generally, to check the robustness of policy analysis under RE future work should study policy under more plausible behavioural learning equilibria.

 

\vskip1cm    


\clearpage